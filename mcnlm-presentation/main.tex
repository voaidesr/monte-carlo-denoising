\documentclass{beamer}

% Theme selection - keeping it clean and professional
\usetheme{Madrid}
\usecolortheme{default}

% Packages
\usepackage{amsmath}
\usepackage{graphicx}
\usepackage{booktabs}

% Title Information
\title[Monte Carlo NLM]{Monte Carlo Optimization of \\ Non-Local Means Denoising Algorithm}
\author[Robu-Verzotti-Voaideș]{Petru-Răzvan Robu, Matteo-Alexandru Verzotti, \\ Robert-Ionuț Voaideș-Negustor}
\date{\today}

\begin{document}

%--- Slide 1: Title ---
\begin{frame}
    \titlepage
\end{frame}

%--- Slide 2: Introduction ---
\begin{frame}{Introduction: The Noise Problem}
    \begin{itemize}
        \item \textbf{Image Noise:} Modeled as additive white Gaussian noise.
        \[ y = x + \eta, \quad \eta \sim \mathcal{N}(0, \sigma) \]
        \item \textbf{Traditional Techniques:}
            \begin{itemize}
                \item Gaussian smoothing, Median filtering.
                \item \textit{Issue:} Relies on locality, causing blurred edges and lost textures.
            \end{itemize}
        \item \textbf{Non-Local Means (NLM):}
            \begin{itemize}
                \item Preserves texture by using similar patches from \textit{anywhere} in the image.
                \item \textit{Issue:} High computational complexity.
            \end{itemize}
    \end{itemize}
\end{frame}

%--- Slide 3: The NLM Algorithm ---
\begin{frame}{Non-Local Means (NLM) Formulation}
    The denoised value $z(p)$ is a weighted average of pixels $y(q)$:
    
    \[ z(p) = \frac{1}{C(p)} \sum_{q \in \Omega} w(p,q) y(q) \]
    
    \begin{itemize}
        \item \textbf{Weights ($w$):} Measure similarity between patches centered at $p$ and $q$.
        \[ w_i = e^{-||y - x_i||^2 / (2h_r^2)} \]
        \item \textbf{Complexity:} $\mathcal{O}(m n d)$
            \begin{itemize}
                \item $m$: Pixels to denoise
                \item $n$: Number of patches (search space)
                \item $d$: Patch dimensions
            \end{itemize}
    \end{itemize}
\end{frame}

\begin{frame}{NLM Denoising Results}
    \centering
    % Replace 'filename' with your actual image file name (e.g., comparison.png)
    \includegraphics[width=0.75\textwidth]{./res/nlm_denoise2.pdf}
    
    \vspace{0.5cm} % Adds a little space between image and text
\end{frame}

%--- Slide 4: Monte Carlo Optimization ---
\begin{frame}{Monte Carlo Non-Local Means (MCNLM)}
    \begin{block}{The Goal}
    Reduce complexity from $\mathcal{O}(mnd)$ to $\mathcal{O}(mkd)$ where $k \ll n$.
    \end{block}

    \textbf{Core Strategy:}
    \begin{itemize}
        \item Instead of computing weights for \textit{all} $n$ patches, select a random subset $k$.
        \item Use Bernoulli sampling to generate a reference set.
        \item \textbf{Sampling Pattern $p$:} Probability vector determining if a patch is selected.
    \end{itemize}
\end{frame}

%--- Slide 5: Stochastic Approximation ---
\begin{frame}{Mathematical Approximation}
    We approximate the NLM fraction using random variables $A$ (numerator) and $B$ (denominator):

    \[
        A = \frac{1}{n}\sum_{j=1}^{n} w_j x_j \frac{I_j}{p_j}, \quad 
        B = \frac{1}{n}\sum_{j=1}^{n} w_j \frac{I_j}{p_j}
    \]

    \begin{itemize}
        \item $I_j \sim \text{Bernoulli}(p_j)$: Indicator variable for sampling.
        \item $Z = A/B$ is the estimator for the true pixel value $z$.
        \item \textbf{Note:} $Z$ is a \textit{biased} estimator, but the error drops exponentially as sample size increases.
    \end{itemize}
\end{frame}

%--- Slide 6: Spatial Sampling ---
\begin{frame}{Improving Spatial Locality}
    Pure NLM ignores spatial distance. We reintroduce locality for better results.
    
    \textbf{Combined Weight Function:}
    \[ w_j = w_j^r \cdot w_j^s \]
    
    \begin{itemize}
        \item $w_j^r$: Structural similarity (standard NLM).
        \item $w_j^s$: Spatial proximity.
        \[ w_j^s = e^{-(d_2^j)^2 / (2h_k^2)} \cdot \mathbb{I}\{d_{\infty}^j \le \rho\} \]
        \item Helps distinguish features (e.g., a bright star vs. noise) based on surroundings.
    \end{itemize}
\end{frame}

\begin{frame}{Non-local matches}
    \begin{columns}[c] % [c] vertically centers the images
        
        % First Column
        \column{0.48\textwidth}
        \centering
        \includegraphics[width=\linewidth]{./res/mc_matches_1.pdf}
        
        % Second Column
        \column{0.48\textwidth}
        \centering
        \includegraphics[width=\linewidth]{./res/mc_matches_2.pdf}
        
    \end{columns}
    
    \vspace{0.5cm}
    \centering
\end{frame}

%--- Slide 7: Algorithm Logic ---
\begin{frame}{Algorithm: Monte Carlo NLM}
    \begin{enumerate}
        \item \textbf{Input:} Noisy patch $y$, reference set $X$.
        \item \textbf{Loop} through references $j=1 \dots n$:
            \begin{itemize}
                \item Generate random variable $I_j \sim \text{Bernoulli}(p_j)$.
                \item \textbf{If} $I_j = 1$ (Sampled):
                \item Compute weight $w_j$.
                \item Update sums for Numerator ($A$) and Denominator ($B$).
            \end{itemize}
        \item \textbf{Output:} $Z = A/B$.
    \end{enumerate}
\end{frame}

\begin{frame}{MCNLM Results}
    \centering
    % Replace 'filename' with your actual image file name (e.g., comparison.png)
    \includegraphics[width=1.01\textwidth]{./res/mcnlm3.pdf}
    
    \vspace{0.5cm} % Adds a little space between image and text
\end{frame}

\begin{frame}{MCNLM Results}
    \centering
    % Replace 'filename' with your actual image file name (e.g., comparison.png)
    \includegraphics[width=1.01\textwidth]{./res/mcnlm1.pdf}
    
    \vspace{0.5cm} % Adds a little space between image and text
\end{frame}



%--- Slide 8: Theoretical Bounds ---
\begin{frame}{Why it Works: Error Bounds}
    Does the approximation converge?
    
    \begin{block}{Exponential Error Decay}
    For sample size $n$ and error tolerance $\epsilon$:
    \[ \mathbb{P}(|Z - z| > \epsilon) \le \text{Exponential Decay Terms} \]
    \end{block}

    \begin{itemize}
        \item \textbf{Empirical Proof:} Even with only \textbf{5\%} of samples ($p=0.05$):
        \[ \mathbb{P}(\text{Error} > 0.01) \le 6.5 \times 10^{-6} \]
        \item High reliability with fraction of the cost.
    \end{itemize}
\end{frame}

%--- Slide 9: Experimental Results ---
\begin{frame}{Experimental Results}
    \begin{columns}
        \column{0.5\textwidth}
        \textbf{MSE Convergence:}
        \begin{itemize}
            \item MSE drops rapidly as sampling ratio increases.
            \item Diminishing returns after $p=0.3$.
        \end{itemize}
        
        \column{0.5\textwidth}
        \textbf{Visual Quality:}
        \begin{itemize}
            \item Preserves edges better than Gaussian.
            \item Comparable visual quality to full NLM.
        \end{itemize}
    \end{columns}

    \vspace{1cm}
    \textit{(Placeholder for Figure 4/6: MSE Comparison Charts)}
\end{frame}

\begin{frame}{Comparison Results}
    \begin{columns}[c] % [c] vertically centers the images
        
        % First Column
        \column{0.48\textwidth}
        \centering
        \includegraphics[width=\linewidth]{./res/convergence1_mse.pdf}
        
        % Second Column
        \column{0.48\textwidth}
        \centering
        \includegraphics[width=\linewidth]{./res/convergence1_psnr.pdf}
        
    \end{columns}
    
    \vspace{0.5cm}
    \centering
    \textit{Figure: Visual comparison showing MSE and PSNR convergence.}
\end{frame}

\begin{frame}{Increasing window size}
    \centering
    % Replace 'filename' with your actual image file name (e.g., comparison.png)
    \includegraphics[width=0.75\textwidth]{./res/window_size_mse.pdf}
    
    \vspace{0.5cm} % Adds a little space between image and text
    
    \textit{Figure: Plotting MSE as window size is increased}
\end{frame}

%--- Slide 10: Handling Unknown Noise ---
\begin{frame}{Noise Estimation (FFT)}
    In real applications, noise deviation ($\sigma$) is unknown.
    
    \textbf{Solution: Fast Fourier Transform (FFT)}
    \begin{enumerate}
        \item Convert image to Frequency Domain.
        \item High-frequency components $\approx$ Noise.
        \item Mask low frequencies and compute standard deviation of the remainder.
    \end{enumerate}
    
    \begin{itemize}
        \item \textbf{Result:} Estimated $\sigma \approx 20.33$ vs True $\sigma = 17$.
        \item Sufficient for effective denoising.
    \end{itemize}
\end{frame}

\begin{frame}{Noise Comparison}
    \centering
    % Replace 'filename' with your actual image file name (e.g., comparison.png)
    \includegraphics[width=1.01\textwidth]{./res/noise_comparison_visual.pdf}
    
    \vspace{0.5cm} % Adds a little space between image and text
\end{frame}



%--- Slide 11: Conclusion ---
\begin{frame}{Conclusion}
    \begin{itemize}
        \item \textbf{Efficiency:} MCNLM significantly reduces computational cost ($\mathcal{O}(mkd)$).
        \item \textbf{Quality:} Maintains high Peak Signal-to-Noise Ratio (PSNR).
        \item \textbf{Robustness:} Theoretical bounds prove reliability even with low sampling rates (e.g., 5-10\%).
        \item \textbf{Practicality:} Can be combined with spatial weights and FFT noise estimation for real-world usage.
    \end{itemize}
    
    \vspace{0.5cm}
    \centering \Large \textbf{Thank You!}
\end{frame}

\end{document}